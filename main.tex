\documentclass{beamer}
\usetheme{CambridgeUS}

\title{Assignment 8: Papoullis Text Book}
\author{Cherukupalli Sai Malini Mouktika}
\date{\today}
\logo{\large \LaTeX{}}

\usepackage{graphicx}
\usepackage{amsmath}
\usepackage{amssymb}
 \usepackage{listings}
    \usepackage{color}                                            
    \usepackage{array}                                            
    \usepackage{longtable}                                        
    \usepackage{calc}                                            
    \usepackage{multirow}                                         
    \usepackage{hhline}                                          
    \usepackage{ifthen}   

\usepackage{multicolrule}


   
%table commands   
\def\inputGnumericTable{}

\usepackage[latin1]{inputenc}                                 
\usepackage{caption} 
\captionsetup[table]{skip=3pt}  

\renewcommand{\thefigure}{\arabic{table}}
\renewcommand{\thetable}{\arabic{table}}       


\providecommand{\brak}[1]{\ensuremath{\left(#1\right)}}
\renewcommand\thesection{\arabic{section}}
\renewcommand\thesubsection{\thesection.\arabic{subsection}}
\renewcommand\thesubsubsection{\thesubsection.\arabic{subsubsection}}

\newcommand{\graph}{\noindent \textbf{Graph: }}
\newcommand{\calc}{\noindent \textbf{Calculations: }}
\numberwithin{equation}{subsection}

\renewcommand{\thetable}{\theenumi}
\usepackage{amsmath}
\setbeamertemplate{caption}[numbered]{}
\providecommand{\pr}[1]{\ensuremath{\Pr\left(#1\right)}}
\providecommand{\cbrak}[1]{\ensuremath{\left\{#1\right\}}}

\begin{document}

\begin{frame}
    \titlepage 
\end{frame}

\logo{}

\begin{frame}{Outline}
    \tableofcontents
\end{frame}
\section{Question}
\begin{frame}{Question}
    \begin{block}{Example 12.7}
Consider the process 
\begin{align}
y(t) = ax(t) \\   E\lbrace a\rbrace = 0
\end{align}  
where x(t) is a mean-ergodic process independent of the random variable a. Find whether y(t) is mean ergodic or not.
\end{block}
\end{frame}
\section{solution}
\begin{frame}{solution}
Clearly $E\lbrace y(t)\rbrace = 0$ 
\begin{align}
R_{yy}(\tau) = E\lbrace a^{2}x(t+\tau)x(t)\rbrace = \sigma_{a}^{2}R_{xx}(\tau)\\
\end{align}
The spectrum x(t) equals $S_{xx}^{c}(\omega) + 2\pi \eta _{x}^{2}\delta (\omega)$. Hence
\begin{align}
    S_{yy}(\omega) = \sigma_{a}^{2}S_{xx}^{c}(\omega) + 2\pi \sigma_{a}^{2} \eta _{x}^{2}\delta (\omega)
\end{align}
This shows that the process y(t) is not mean-ergodic. 
\end{frame}
\end{document}

